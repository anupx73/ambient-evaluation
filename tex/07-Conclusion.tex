\section{Conclusion}
The sidecar-less model of ambient mesh shows some visible performance benefits over sidecar model. Essentially all of this saves cloud cost and provides better operational excellence. Looking at the research results and after answering the research questions successfully, ambient mesh can be seen as the future of service mesh while it is backed by Linux kernel technology. It is also beneficial for application developers and platform engineering teams to consume or manage Istio in a more efficient way. It is expected that the architecture of ambient mesh will evolve with its stable releases and more performance benefits will be seen. Ambient mode may become the default installation mode for Istioctl in future (\cite{ambientExplained}) however shifting from sidecar architecture will take some time based on its popularity and current footprints. Ambient mesh is currently a part of commercial service mesh project Gloo (\cite{glooMeshIntro}) and in near future more companies may adopt to its architecture.

\subsection{Future Work}
Ambient mesh being the state-of-art technology and Istio's ambient mode released as alpha version at present, an immense opportunity is formulated for future research. Grey literature mentions about Istio’s network latency and identifies the source as intensive layer 7 processing (\cite{istioHoward2022}). As the Waypoint proxy is deployed in a random node, traffic request has to come through multiple hops via local or remote Ztunnel. Future research can also focus on some previous Istio limitation like not supporting Kubernetes Jobs or  server-send-first protocols.