\section{Conclusion}
The sidecar-less model of ambient mesh shows some visible performance benefits over sidecar model. Essentially all of this saves cloud cost and resolves complexities for any business. When it comes to cost savings, many businesses will want to explore this technology in near future. Looking at the research results and after answering the research questions successfully, it can be said that ambient mode of Istio or any other commercial implementation like Gloo mesh (\cite{glooMeshIntro}) can be benefitial for application developers, platform engineering team and ultimaately the business behind. It is expected that the architecture of ambient mesh will evolve with stable releases and more performance benefits can be seen in future researches. Though ambient mesh is offering so many benefits, it may not be possible for the industry to move away from sidecar oriented deployments immediately. Some scenarios like legacy deployments, lack of engineering resource availability to learn new technology or some corner case like regulatory requirements of having all Kubernetes pod containers in same region may restrict some organization to use sidecar model of service mesh even for a prolong time period. But with the rapid growth of cloud native space and ever changing technology it is better to shift towards ambient mesh for any fresh Kubernete deployment where service mesh is required provided that Istio ambient mesh or any other commercial version of ambient mesh is released with production grade quality.

\subsection{Future Work}
Ambient mesh being the state-of-art technology and Istio's ambient mode released as alpha version at present, an immense opportunity is formulated for future scholars. Grey literature mentions about Istio’s network latency and identifies the source as intensive layer 7 processing (\cite{istioHoward2022}). In ambient mesh, the layer 7 or Waypoint proxies can be deployed to different nodes by the Istio system. In this scenario Ztunnel proxies deployed on each node needs to pass traffic through a Waypoint deployed possibly on a separate node. This may lead to add some delay in receiving microservice response. Istio authors claim this delay or latency to be close Istio sidecar implementation (\cite{istioHoward2022}). Future research can focus on this parameter of ambient mesh to verify latency benchmarks of sidecar and ambient mode. This latency test will also decide whether separating proxy layers among L4 and L7 in ambient mode results in some performance compromise. Also historically Istio had some limitation like not supporting Kubernetes Jobs, server-send-first protocols, or a list of reserved ports which could be trivial for some application teams. By introduction of ambient mode in Istio, these could be the case of past times and can be explored via future researches.