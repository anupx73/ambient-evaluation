\section*{Abstract}
Service oriented architecture is at the bleeding edge today and the technologies around it keeps evolving including the service mesh in Kubernetes environment. There are number of successful service mesh products used in current technology space and one of the pioneer is Istio, a Cloud Native Foundation project initially developed by Google and IBM in association with Lyft, the Envoy proxy development team. The use of Envoy proxy in Istio has already made its mark few years back and the sidecar injection implementation is battle tested. The standard of sidecar injection is solid and well tested but practitioners always wanted to improve the resource utilization side of it, hence a new architecture pattern of Istio is born as Ambient mesh. This architecture provides the service mesh capabilities without the sidecar injection in microservice pods and claims to provide better performance without any security compromises. With the release of ambient mesh alpha version, the potential of this architecture is explored in grey literature but no academic paper is produced. This paper explores the performance and complexity aspect of this new service mesh architecture by using Istio in sidecar and ambient mode on \acrfull{gcp}. Multiple tests covering a variety of deployment models used in industry reveals significant improvement in memory consumption of Istio system in ambient mode. On the complexity side, Istio sidecar architecture provides an imperative way to upgrade the Istio system even if the blue-green deployment, Istio revision and tagging features are leveraged. However, ambient mode gives a true declarative way of upgrading Istio system by supporting GitOps principles and provides the ability to segregate the platform engineering and application development team responsibility in managing Istio. The research concludes a shift towards ambient mesh is significant over sidecar mode provided a stable version of ambient mesh is released as part of Istio.