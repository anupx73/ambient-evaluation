\section*{Abstract}
Over the last few years, service mesh has become an essential part of the cloud native technology stack. A service mesh is a dedicated infrastructure layer in the Kubernetes cluster to make service-to-service communication safer, faster, and more reliable. A number of well established service mesh products are used today, including Istio, a project developed by Google, IBM, and Lyft, the company behind Envoy Proxy. Istio is made out of a two-tier architecture that includes a control plane for mesh configuration and a data plane to work as a sidecar container in microservice pods to provide mesh capabilities. The sidecar container uses an Envoy proxy, which requires additional compute resources to run and impacts the Istio system's efficiency in many situations. Hence, a new data plane architecture in Istio was born as ambient mesh. This new data plane provides service mesh capabilities without sidecar injection and claims to achieve greater efficiencies. With the release of the ambient mesh alpha version as part of Istio, the potential of this sidecar-less architecture is explored in gray literature, but no academic paper is produced. The aim of this paper is to evaluate the resource utilization and operational experience improvements of ambient mesh by comparing Istio sidecar and ambient modes on the Google Cloud Platform. Multiple test scenarios covering a variety of deployment models show significant improvements in the resource consumption of the Istio system in ambient mode. The operational experience is also improved in Istio ambient mode by having a declarative installation and upgrade method that does not require a microservice pod restart. Additionally, the segregated data plane layers of ambient mesh help in gradual service mesh adoption. Finally, the research concludes that a shift towards this new data plane architecture is significant, provided a stable version of ambient mesh is released, which will open further research avenues.