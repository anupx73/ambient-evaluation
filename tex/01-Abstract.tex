\section*{Abstract}
Over the last few years service mesh has become an essential part of the cloud native stack. A service mesh is a dedicated infrastructure layer in Kubernetes cluster to make service-to-service communication safer, faster, and reliable. There are number of established service mesh products used today including Istio, a project developed by Google, IBM and Lyft, the company behind envoy proxy. Istio provides a two tier architecture consisting a control and data plane. The data plane is implemented by injecting envoy proxies as a sidecar container to microservice pods to deliver service mesh capabilities. This sidecar containers require additional compute resources to run and impacts the Istio system efficiency in many situations. Hence, a new data plane architecture in Istio was born as Ambient mesh. The new data plane provides service mesh capabilities without sidecar injection and claims to achieve greater efficiencies. With the release of ambient mesh alpha version as part of Istio, the potential of this sidecar-less architecture is explored in gray literature, but no academic paper is produced. The aim of this paper is to evaluate the resource utilization and operational experience improvement of ambient mesh by comparing Istio sidecar and ambient modes on Google Cloud Platform. Multiple test scenarios covering a variety of deployment models shows a significant improvements in resource consumption of Istio system in ambient mode. Operational experience is also improved in Istio ambient mode by having a declarative method for installation and upgrade which does not require a microservice pod restart. Additionally the segregated data plane layers of ambient mesh helps in gradual service mesh adoption. Finally the research concludes that a shift towards this new architecture is significant provided a stable version of ambient mesh is released which will open further research avenues.