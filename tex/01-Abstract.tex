\section*{Abstract}
The rapid development of cloud native technologies has brought many changes to service-oriented architecture and how it is consumed at the infrastructure and application levels. Service mesh in Kubernetes is one such technology that has emerged in the last few years to help development teams focus on business logic instead of thinking about inter-service communication and security practices. There are a number of successful service mesh products used in the current technology space, including Istio, a Cloud Native Foundation project initially developed by Google and IBM in association with Lyft, the Envoy proxy development team. Envoy proxy is injected as a sidecar container to microservice pods to deliver service mesh functionality and this architecture is well adopted in the industry. But Istio practitioners always wanted to improve the resource utilization side of this model, hence, a new architecture pattern was born as Ambient mesh. This new architecture provides service mesh capabilities without sidecar injection and claims to provide better performance without any security compromises. With the release of the ambient mesh alpha version, the potential of this architecture is explored in gray literature, but no academic paper is produced. The aim of this paper is to explore the performance and complexity aspects of this new service mesh architecture by using Istio in sidecar and ambient mode on \acrfull{gcp}. Multiple tests covering a variety of deployment models used in industry reveal significant improvements in the memory consumption of the Istio system in ambient mode. On the complexity side, the Istio sidecar architecture provides an imperative way to upgrade the Istio system, even if the blue-green deployment, Istio revision, and tagging features are leveraged. However, ambient mode gives a true declarative way of upgrading the Istio system by supporting GitOps principles and providing the ability to segregate the platform engineering and application development team's responsibility in managing Istio. The research concludes that a shift towards ambient mesh is significant over sidecar mode, provided a stable version of ambient mesh is released as part of Istio.