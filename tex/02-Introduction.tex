\section{Introduction}
\subsection{Service Mesh and Istio}
When the industry moved from monolith to microservice architecture a decade ago, additional challenges were introduced. Some of the most prominent are service discovery, secure communication between microservices, policy enforcement, and monitoring multi-fold microservices within a single application boundary. All these non-functional requirements easily became an overhead for many development teams, and the service mesh concept was introduced as mitigation path. Istio is one of the most popular implementation of such concept which is heavily used in current cloud enabled businesses.

At the core of Istio, it uses sidecar proxies, which encapsulate all the methods to implement the non-functional requirement without microservice code being impacted (\cite{techcrunch2022}). The sidecar sits as a separate container in a microservice host or Kubernetes pod to intercept all the traffic requests and establish communication, network policy, and security practices that the business expects from a service out of the box. In this architecture, additional containers are deployed alongside service containers; hence, it increases the resource utilization and operational complexity of the application. Despite these challenges, the functionality and reliability these sidecars provide are incredible.

\subsection{Motivation}
Performance optimization remains an engineering task for any software team, and service mesh is no different. There are several pieces of literature published on comparing Istio and other service meshes from a performance and security perspective, but there is no paper that describes a different architecture for service mesh altogether. Most of the service meshes available today either use Envoy or in-house-developed sidecar proxies, which determines the success factor of the product. A limited effort is published on the Internet to invent a solution from the ground up to break this practice. This paper explores one such technology by identifying a new architecture and evaluating it in a Kubernetes environment.

\subsection{Research Questions and Objectives}
The following questions are formulated for the purpose of this research: 

\textbf{RQ1. Does Istio ambient mode consumes less compute resources than sidecar mode?}

\textbf{RQ2. Does the sidecar-less architecture reduces operational complexities?}

To find the answer to the above research questions, a systemic methodology is followed by setting up the infrastructure for running Istio and supporting components on a Kubernetes cluster. Firstly, the objective of this paper is to investigate the resource utilization of the Istio system in sidecar and ambient modes in different Kubernetes environments. Secondly, an effort is made to further dig down into the Istio system to find out the operational complexity of two different Istio modes and compare them from an operator or platform engineering perspective.
