\section{Discussion}
The sidecar less design used by Istio Ambient mesh is transparent to the workloads within the mesh. Numerous advantages come with this strategy, such as enhanced operations, gradual acceptance, and more [22]. This section evaluates the ambient mesh offerings over Istio's sidecar mode based on the research result data revealed in section 4 along with addressing the research questions.

RQ1. Does the ambient mesh consumes less compute resources than Istio's sidecar mode and performs better?


RQ2. Does the sidecar less architecture reduces operational complexities like rolling out and Istio system upgradability?

Use: https://github.com/istio/istio/tree/master/samples/ambient-argo

When it comes to cluster administration, a small organization gets some advantages of restarting their workloads while the active traffic is not very high. This remains valid for any size of organizations when the number of microservices are limited and its fairly easy to upgrade Kubernetes system components by a few minutes of downtime of the system. But in case of a large organization, these scenario changes by a large margin and the Service Level Objective (SLO) needs to be kept in mind. When the number of microservices are hundreds or in thousands and managed by several different teams, Kubernetes administrators has to make sure minimum downtime while making any changes to Kubernetes system components including Istio. When the Kubernetes down time increases Site Reliability Engineering (SRE) team also needs to adjust all the alerts and notify teams about any false alarms. So, any change in Kubernetes infrastructure like Istio version upgrade needs involvement from multiple functional teams. Ambient mesh being non intrusive to the pods, all these Kubernetes system level changes remains unnoticed to the pods. Some features of service mesh may not be available during the maintenance time, but the overall complexity is reduced. This helps the Kubernetes administrators to keep all the controls in their hands instead of involving other stakeholders like SRE or Application teams. Though this Istio system upgrade is not a daily job but in a high traffic enabled application or business critical applications this becomes crucial. Even when an organization operates from cross geographic location, this can play a vital  role to keep the toil minimum as it becomes really difficult to co-ordinates with people from different time zone for any system maintenance job. All these are essentially possible due to having ztunnel installed at node level and ip-tables rules configured at node level by eBPF technology.