\section{Discussion}
The sidecar less design used by Istio Ambient mesh is transparent to the workloads within the mesh. Numerous advantages come with this strategy, such as enhanced operations, gradual acceptance, and more. This section evaluates the ambient mesh offerings over Istio's sidecar mode based on the research result data revealed in Section \ref{resultSection} along with addressing the research questions.

RQ1. Does the ambient mesh consumes less compute resources than Istio sidecar mode?

With multiple performance tests with single namespace and multiple namespace microservice deployment, ambient mode performance is significantly appreciable. Looking at the numbers, the pod memory utilization is improved by 23\% in ambient mode and comparing the Istio system memory consumption, it is improved by [TBD] over sidecar mode. Also, there is a CPU utilization optimization seen as well which is at [TBD]. Hence, ambient mesh brings performance improvement and as the tests performed in this research covered multiple deployment models the outcome of this research becomes more prominent and reliable.

RQ2. Does the sidecar less architecture reduces operational complexities in Istio system upgrade and installation?

When it comes to cluster administration, a small organization gets some advantages of restarting their workloads while the active traffic is not very high. This remains valid for any size of organizations when the number of microservices are limited and its fairly easy to upgrade Kubernetes system components by a few minutes of downtime of the system. But in case of a large organization, these scenario changes by a large margin and the \acrlong{slo} needs to be kept in mind. When the number of microservices are hundreds or thousands and managed by several different teams, Kubernetes administrators has to make sure minimum downtime while making any changes to Kubernetes system components including Istio. When the Kubernetes downtime increases \acrlong{sre} team also needs to adjust all the alerts and notify teams about any false alarms. So, any change in Kubernetes infrastructure like Istio version upgrade needs involvement from multiple functional teams. Ambient mesh being non intrusive to the pods, all these Kubernetes system level changes remains unnoticed to the pods. Some features of service mesh may not be available during the maintenance time, but the overall complexity is reduced. This helps the Kubernetes administrators to keep all the controls in their hands instead of involving other stakeholders like \acrshort{sre} or Application teams. Though this Istio system upgrade is not a regular task but in a high traffic enabled application or business critical applications this becomes crucial. Even when an organization operates from cross geographic location, this can play a vital  role to keep the toil minimum as it becomes really difficult to co-ordinates with people from different time zone for any system maintenance job. All these are essentially possible due to having ztunnel installed at node level and ip-tables rules configured at node level by eBPF technology.

Also it is important to note that Istio sidecars break the declarative principle since the version of the injector used when the pod was built determines the version of Istio injected in the sidecar at runtime (\cite{gitMitch2023}). The sidecar does not upgrade when the injector is upgraded rather, all injected pods must be restarted for their sidecars to upgrade. This is an imperative action as opposed to a declarative one violating the GitOps principles (\cite{gitopsBook}). On a live Kubernetes cluster with hundreds or thousands of microservices, upgrading Istio becomes a challenge for this imperative nature of current Istio implementation. To minimize the impact, Istio documentation (\cite{istioDocCanaryUpgrade}) recommends to use blue-green and canary deployment strategies to roll out new versions of Istiod control plane and ingress controller respectively, but still pod restart is inevitable. With ambient mode this restriction is expected to go away as it does not rely on sidecars. A test is performed with a Kubernetes load balancer behind the Istio ingress gateways to apply the blue-green deployment model to upgrade the Istio version.
