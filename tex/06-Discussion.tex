\section{Discussion}
This section emphasizes the ambient mesh offerings over sidecar architecture based on the research findings by comparing Istio sidecar and ambient mode. In Kubernetes environment, service mesh like Istio provides request routing, request filtering, rate limiting, circuit breaking and many more features to ensure a secured and feature rich mesh experience without any additional change in microservices. All these features comes with certain performance penalties and at enterprise level with growing number of microservices, this becomes an important area for improvement. Where sidecar oriented service meshes provide a standard way of implementing service mesh today, the sidecar less design used by ambient mesh gives a transparent integration of service mesh to the system with less resources utilization. Apart from less resources consumption, numerous advantages come with this strategy such as enhanced operations, gradual acceptance and more. Lets discuss further on this by answering the research questions.

\textbf{RQ1. Does the ambient mesh consumes less compute resources than Istio sidecar mode?}

With multiple tests in single namespace and multiple namespace environments, ambient mode performance is significantly appreciable. Looking at the numbers, the pod memory utilization is improved by 23\% in ambient mode and comparing the Istio system memory consumption, it is improved by more than 70\% over sidecar mode. Looking at the exact data for comparing sidecar mode against ambient mode only with layer 4 processing, the memory consumption is 88.47\% and 95.13\% (Table \ref{res:sidecarMemVsL4}) better in single and multiple namespaces respectively. While comparing the results between sidecar and ambient mode with layer 4 and layer 7 processing the difference is 79.45\% and 74.98\% (Table \ref{res:sidecarMemVsL4L7}) in single and multiple namespaces. CPU resource utilization is also improved in ambient mode as the result shows an improvement of 51.4\% and 48\% for single and multiple namespaces while comparing sidecar with ambient layer 4 processing. Finally when sidecar mode is tested against ambient mode with layer 4 and layer 7 processing enabled, CPU improvement percentage is seen as 7.4\% and 39\% for single and multiple namespaces. Based on these results including the comparison percentages, ambient mode definitely brings some significant savings in compute resources over sidecar mode.

\textbf{RQ2. Does the sidecar less architecture reduces operational complexities in Istio system upgrade and installation?}

As ambient mode does not rely on sidecar injection, in case of Istio system installation and upgrade, pod restart is not required. On the Istio upgrade side, there is no significant difference is noticed as using blue-green deployment model the service downtime remains nominal in both sidecar and ambient modes. When comparing operational complexities another relevant thing to note is, the gradual roll out of service mesh in ambient mode. This allows an organization to try out the default layer 4 processing of ambient mode before deep diving into layer 7 processing. So, to answer the research question, ambient mode does reduces the operational complexities including opening some additional operational benefits to its users.

When it comes to cluster administration, a small organization gets some advantages of restarting their workloads while the active traffic is not very high. This remains valid for any size of organizations when the number of microservices are limited and its fairly easy to upgrade Kubernetes system components by a few minutes of downtime of the system. But in case of a large organization, these scenario changes by a large margin and the \acrlong{slo} needs to be managed. When the number of microservices are hundreds or thousands and managed by several different teams, Kubernetes administrators has to make sure minimum downtime while making any changes to Kubernetes system components including Istio. When the Kubernetes downtime increases \acrlong{sre} team also needs to adjust all the alerts and notify teams about any false alarms. So, any change in Kubernetes infrastructure like Istio version upgrade needs involvement from multiple functional teams. Ambient mesh being non intrusive to the pods, all these Kubernetes system level changes remains unnoticed to the pods. Some features of service mesh may not be available during the maintenance time, but the overall complexity is reduced. This helps the Kubernetes administrators to keep all the controls in their hands instead of involving other stakeholders like \acrshort{sre} or Application teams. Though the Istio system upgrade is not a regular task, but for an enterprise grade ecosystem this becomes crucial when Istio version needs to be patched due to any vulnerabilities. Even when an organization operates from cross geographic location, this can play a vital  role to keep the toil minimum as it becomes really difficult to co-ordinates with people from different time zone for any system maintenance job. All these are essentially possible due to having ztunnel installed at node level and ip-tables rules configured at node level by eBPF technology avoiding pod restart. Also it is important to note that Istio sidecars break the declarative principle since the version of the injector used when the pod was built determines the version of Istio injected in the sidecar at runtime (\cite{gitMitch2023}). The sidecar does not upgrade when the injector is upgraded rather, all injected pods must be restarted for their sidecars to upgrade. This is an imperative action as opposed to a declarative one violating the GitOps principles (\cite{gitopsBook}). On a live Kubernetes cluster with hundreds or thousands of microservices, upgrading Istio becomes a challenge for this imperative nature of current Istio implementation. To minimize the impact, Istio documentation (\cite{istioDocCanaryUpgrade}) recommends to use blue-green and canary deployment strategies to roll out new versions of Istiod control plane and ingress controller respectively, but still pod restart is inevitable. With ambient mode this restriction is expected to go away as it does not rely on sidecars. A test is performed with a Kubernetes load balancer behind the Istio ingress gateways to apply the blue-green deployment model to upgrade the Istio version. Finally, before closing the discussion, it is worth to note that the tests performed in this research covers a wide variety of deployment scenarios, hence, the outcome becomes more reliable and applicable to industry grade use cases.
